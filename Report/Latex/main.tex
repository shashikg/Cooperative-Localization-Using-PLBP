\documentclass[12pt]{article}
\usepackage[english]{babel}
\usepackage{natbib}
\usepackage{url}
\usepackage[utf8x]{inputenc}
\usepackage{amsmath}
\usepackage{graphicx}
\graphicspath{{images/}}
\usepackage{parskip}
\usepackage{fancyhdr}
\usepackage{vmargin}
\usepackage{subfig}
\usepackage{tikz}
\usepackage[normalem]{ulem}
\usepackage{amssymb}
\usetikzlibrary{shapes.geometric, arrows}
\tikzstyle{startstop} = [rectangle, rounded corners, minimum width=2cm, minimum height=0.5cm,text centered, draw=black, fill=red!30]
\tikzstyle{io} = [trapezium, trapezium left angle=70, trapezium right angle=110, minimum width=3cm, minimum height=0.5cm, text centered, draw=black, fill=blue!30]
\tikzstyle{process} = [rectangle, minimum width=3cm, minimum height=0.5cm, text centered, draw=black, fill=orange!30]
\tikzstyle{decision} = [diamond, minimum width=2cm, minimum height=0.5cm, text centered, draw=black, fill=green!30]
\tikzstyle{arrow} = [thick,->,>=stealth]
\usetikzlibrary{positioning,fit,calc}
\tikzset{block/.style={draw,thick,text width=2cm,minimum height=1cm,align=center},
         line/.style={-latex}
}

\setmarginsrb{3 cm}{2.5 cm}{3 cm}{2.5 cm}{1 cm}{1.5 cm}{1 cm}{1.5 cm}

\title{Cooperative Localization Using Posterior Linearization Belief Propagation}								% Title
\author{SHIVRAM MEENA\\ 
		SHASHI KANT GUPTA \\
		MAMILLA SIVASANKAR \\
		PRADEEP KUMAR \\
		ALLAPARTHI VENKATA SATYA VITHIN \\}								% Author
\date{\today}											% Date

\makeatletter
\let\thetitle\@title
\let\theauthor\@author
\let\thedate\@date
\makeatother

\pagestyle{fancy}
\fancyhf{}

\rhead{Group-9}
\cfoot{\thepage}

\begin{document}

%%%%%%%%%%%%%%%%%%%%%%%%%%%%%%%%%%%%%%%%%%%%%%%%%%%%%%%%%%%%%%%%%%%%%%%%%%%%%%%%%%%%%%%%%

\begin{titlepage}
	\centering
    \vspace*{0.5 cm}

	\textsc{\Large Term Paper}\\[0.5 cm]				% Course Code
					% Course Name
	\rule{\linewidth}{0.2 mm} \\[0.4 cm]
	{ \huge  \thetitle}\\
	\rule{\linewidth}{0.2 mm} \\[4 cm]
	\textsc{\large Group-9}\\[0.5 cm]
	\begin{minipage}{0.7\textwidth}
		\begin{flushleft} \large
			\emph{Author:}\\
			\theauthor
			\end{flushleft}
			\end{minipage}~
			\begin{minipage}{0.4\textwidth}
			\begin{flushright} \large
			\emph{Student Number:} \\
			150686 \\
160645 \\
17104091 \\
18104074 \\
18104265\\							% Your Student Number
		\end{flushright}
	\end{minipage}\\[2 cm]
	\textsc{\large Statistical Signal Processing EE602A}\\[0.5 cm]
	{\large \thedate}\\[2 cm]
 
	\vfill
	
\end{titlepage}

%%%%%%%%%%%%%%%%%%%%%%%%%%%%%%%%%%%%%%%%%%%%%%%%%%%%%%%%%%%%%%%%%%%%%%%%%%%%%%%%%%%%%%%%%

\tableofcontents
\pagebreak

%%%%%%%%%%%%%%%%%%%%%%%%%%%%%%%%%%%%%%%%%%%%%%%%%%%%%%%%%%%%%%%%%%%%%%%%%%%%%%%%%%%%%%%%%

\section{AIM/OBJECTIVE}
  To infer the positions of the sensor nodes in cooperative fashion using the posterior linearization belief propagation (PLBP) algorithm with nonlinear measurements.
\section{PROBLEM DEFINITION}
In cooperative localization,there are some anchor nodes whose positions are known accurately. The reaming nodes infer their positions based on inter measurements with anchor nodes. The measurements are generally non linear. Hence we linearize the model by using statistical linearization using unsent kalman filtering with sigma points.Belief propagation algorithm will be applied on the linearized posteriors.Strategies which increases the computational efficiency and noise robust are required.
\section{SYSTEM MODEL/METHODOLOGY}
A graph G=(V,E) is formed by a collection of vertices/nodes V=(1,...,m), where m is the number of nodes, and a collection of edges E⊂V×V. Each edge consists of a pair of nodes (i,j) ∈ E. The state of node i is represented by $x_i∈R^{n_x}$\\

We assume $x_{i}$  has Gaussian PDF\\ 
 \begin{equation} 
p_{i}(x_{i}) =\mathcal {N}\left(x_{i};\overline{x}_{i},P_{i}\right)
\end{equation}
With $x_{i}$ representing the state of the node and  $\overline{x}_{i},P_{i}$ are mean and co-variance respectively\\
  
\begin{enumerate}
\item Defining the system model.
	\begin{equation}
z_{i,j}=h_{i,j}(x_{i},x_{j})+n_{i,j}
\end{equation}
$h_{i,j}$ represent the measurement fuction between two nodes i and j. $n_{i,j}$ is the noise measurement. \\

\item Nonlinear measurement function.
	\begin{equation}
	h_{i,j}(x_{i},x_{j})=\sqrt{(p_{x,i}-p_{x,j})^{2}+(p_{y,i}-p_{y,j})^{2}}
	\end{equation}

\item Linearization model of non linear measurements
\begin{equation}
h_{i,j} \approx A_{i,j}^{1}x_{i}+A_{i,j}^{2}x_{j}+b_{i,j}+e_{i,j}
\end{equation}	
\begin{enumerate}
\item Select m sigma points $\chi_{0},\chi_{1},\cdots,\chi_{m}$
\item Propagate sigma points $Z_{j}=h(\chi_{j})$
\item Compute mean and variance.
\begin{equation}
\overline{z}=\sum _{j=1}^{m}\omega _{j}\mathcal {Z}_{j}
\end{equation}
\begin{equation}
\Psi =\sum _{j=1}^{m}\omega _{j}\left(\mathcal
 {X}_{j}-\overline{x}\right)\left(\mathcal {Z}_{j}-\overline{z}\right)^{T}
\end{equation}
\begin{equation}
\Phi =\sum _{j=1}^{m}\omega _{j}\left(\mathcal {Z}_{j}-\overline{z}\right)\left(\mathcal {Z}_{j}-\overline{z}\right)^{T}
\end{equation}
\begin{equation}
\begin{split}
 A^{\;+\;}=&\Psi ^{T}P^{-1} \\
 b^{\;+\;}=&\overline{z}-A^{\;+\;}\overline{x}\\
 \Omega ^{\;+\;}=&\Phi -A^{\;+\;}P\left(A^{\;+\;}\right)^{T}
 \end{split}
\end{equation}
\end{enumerate}


\item Belief propagation on linearized model.
\begin{enumerate}

 \item  The message $ \mu_{i \rightarrow j}$ from node i to j is given as
  \begin{equation}
\mu_{i \rightarrow j} \propto \int l_{i,j}(z_{i,j}|x_{i},x_{j})\mathcal{N} (x_{ i };\bar{x}_{ i },P_{i})\prod_{p \in n(i)\setminus \{ j \} } \mu_{p\rightarrow i}(x_{i}) dx_{i}
\end{equation}
\item under approximation

\begin{equation}
   \mu_{i \rightarrow j}(x_{j}) \propto \mathcal{N} (\alpha_{i \rightarrow j};H_{i \rightarrow j}x_{j},\tau_{i \rightarrow j})
\end{equation}
 \begin{equation}
   \alpha_{i \rightarrow j}=z_{i,j}-A^{1}_{i,j}\bar{x}_{i \rightarrow j}-b_{i,j}
\end{equation} 
 \begin{equation}
 H_{i \rightarrow j}=A^{2}_{i,j}
\end{equation}
 \begin{equation}
 \tau_{i \rightarrow j}=R_{i,j}+\Omega_{i,j}+A^{1}_{i,j}P_{i \rightarrow j}(A^{1}_{i,j})^{T}
\end{equation} 

\end{enumerate}

\end{enumerate}
\pagebreak
\subsection{Algorithm flow}
\begin{figure}[h]


\centering
\subfloat[]{\includegraphics[width=1\textwidth]{v1_1.png}}\\

\caption{ algorithm }
\label{carrdh}
\end{figure}

\section{Matlab simulations}

\begin{align*}
  Area		&	:&  100 m X 100 m\\
Number\;of\; Anchor\; nodes	&	:&   13\\
Number\; of\; Normal \;nodes      	&	:&   100\\
Variance\; of\; Normal\; Nodes	&	:&  100 m\\
Variance\; of\; Anchor\; Nodes	&	:& 0.1 m\\
Number \;of\; Iterations 		&	:&  20\\
Range\; Measurement\; error           &	:&  1 m^2 
\end{align*}

A MATLAB simulation is performed for the above problem according to the system model and generated the following MATLAB functions.\\
%Function name : Acheving_CRLB_Iterative_Estimations_v0.m
Function name : Acheving\_CRLB\_Iterative\_Estimations\_v$0$.m\\
Input:
\begin{enumerate}
\item Number of samples(N)$: 100$ and $n=-50$ to $50$
\item Number of sensor nodes(M)$: 100$ 
\item Noise variance          $: 0.01$ 
\item iterations          $: 10000$ 
\item $\alpha = 1$ and constant across all nodes.          
\item $A_{m} = 0.01 $ to $1$  and random across all nodes.    
\item $\beta_{m} = -1 $ to $1$  and random across all nodes.  
\end{enumerate}
Output: Estimation of bounds for the parameter estimation.\\
Function description:

\section{Results and conclusions}
 \begin{figure}
     \centering
\subfloat[]{\includegraphics[width=0.5\textwidth]{v0_01.png}}\\
\subfloat[]{\includegraphics[width=0.5\textwidth]{v0_02.png}}\\
\subfloat[]{\includegraphics[width=0.5\textwidth]{v0_03.png}}\\
\caption{ Parameter estimation in test sensor system (a) Estimation of $\alpha$ (b)  Estimation of $\alpha_{m}$ (b) Estimation of $\beta_{m}$ }
\label{carrdh}
\end{figure}



\newpage
\section{References}
	\begin{itemize}
		\item[1] H. Wymeersch, J. Lien, and M. Win, “Cooperative localization in wireless
networks,” Proc. IEEE, vol. 97, no. 2, pp. 427–450, Feb. 2009.
		\item[2] S. Kianoush, A. Vizziello, and P. Gamba, “Energy-efficient and mobile-
aided cooperative localization in cognitive radio networks,” IEEE Trans.
Veh. Technol., vol. 65, no. 5, pp. 3450–3461, May 2016.
		\item[3] T. V. Nguyen, Y. Jeong, H. Shin, and M. Z. Win, “Least square cooperative
localization,” IEEE Trans. Veh. Technol., vol. 64, no. 4, pp. 1318–1330,
Apr. 2015.
		\item[4] W. Yuan, N. Wu, B. Etzlinger, H. Wang, and J. Kuang, “Cooperative
joint localization and clock synchronization based on Gaussian message
passing in asynchronous wireless networks,” IEEE Trans. Veh. Technol.,
vol. 65, no. 9, pp. 7258–7273, Sep. 2016.
		\item[5] S. J. Julier and J. K. Uhlmann, “Unscented filtering and non-
linear estimation,” in Proc. IEEE, vol. 92, no. 3, pp. 401–422,
Mar. 2004.
		\item[6] F. Meyer, O. Hlinka, and F. Hlawatsch, “Sigma point belief prop-
agation,” IEEE Signal Process. Lett., vol. 21, no. 2, pp. 145–149,
Feb. 2014.
		\item[7] W. Sun and K.-C. Chang, “Unscented message passing for arbitrary con-
tinuous variables in Bayesian networks,” in Proc. 22nd Nat. Conf. Artif.
Intell., 2007, pp. 1902–1903.
		\item[8] S. Särkkä, Bayesian Filtering and Smoothing. Cambridge, MA, USA:
Cambridge Univ. Press, 2013.

		\item[9] D. Bickson, “Gaussian belief propagation: Theory and application,”
Ph.D. dissertation, The Hebrew Univ. Jerusalem, Jerusalem, Israel,
2008.
		\item[10] Q. Su and Y.-C. Wu, “On convergence conditions of Gaussian belief
propagation,” IEEE Trans. Signal Process., vol. 63, no. 5, pp. 1144–1155,
Mar. 2015.			\item[11] P. Tichavsky, C. H. Muravchik, and A. Nehorai, “Posterior Cramér–Rao
bounds for discrete-time nonlinear filtering,” IEEE Trans. Signal Process.,
vol. 46, no. 5, pp. 1386–1396, May 1998.
		\item[12] S. Li, M. Hedley, and I. B. Collings, “New efficient indoor cooperative
localization algorithm with empirical ranging error model,” IEEE J. Sel.
Areas Commun., vol. 33, no. 7, pp. 1407–1417, Jul. 2015.
		
	\end{itemize}
















\end{document}